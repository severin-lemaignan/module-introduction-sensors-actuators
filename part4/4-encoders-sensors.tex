%!TEX program = xelatex

\documentclass[compress]{beamer}
%--------------------------------------------------------------------------
% Common packages
%--------------------------------------------------------------------------

\definecolor{links}{HTML}{663000}
\hypersetup{colorlinks,linkcolor=,urlcolor=links}

\usepackage[english]{babel}
\usepackage{pgfpages} % required for notes on second screen
\usepackage{graphicx}

\usepackage{multicol}

\usepackage{tabularx,ragged2e}
\usepackage{booktabs}

\setlength{\emergencystretch}{3em}  % prevent overfull lines
\providecommand{\tightlist}{%
  \setlength{\itemsep}{0pt}\setlength{\parskip}{0pt}}


\usetheme{hri}

% Display the navigation bullet even without subsections
\usepackage{remreset}% tiny package containing just the \@removefromreset command
\makeatletter
\@removefromreset{subsection}{section}
\makeatother
\setcounter{subsection}{1}


\newcommand{\source}[2]{{\tiny\it Source: \href{#1}{#2}}}

\usepackage{tikz}
\usetikzlibrary{mindmap,backgrounds,positioning,calc,patterns}
\usepackage{pgfplots}
\pgfplotsset{compat=newest}
\usepackage{circuitikz}

\graphicspath{{part4/figs/}}

\title{ROCO222 \newline Intro to Sensors and Actuators}
\subtitle{Part 4 -- Encoders \& Sensors}

\date{}
\author{Séverin Lemaignan}
\institute{Centre for Neural Systems and Robotics\\{\bf Plymouth University}}

\begin{document}

\licenseframe{github.com/severin-lemaignan/module-mobile-and-humanoid-robots}

\maketitle

\section{Measuring position}

{\fullbackground[scale=0.9,page=2]{ian-sensors.pdf}
    \begin{frame}{Digital incremental encoder}
    \end{frame}
}
{\fullbackground[scale=0.9,page=3]{ian-sensors.pdf}
    \begin{frame}{Optical incremental encoder}
    \end{frame}
}

{\fullbackground[scale=0.9,page=4]{ian-sensors.pdf}
    \begin{frame}{Quadrature output signal generation}
    \end{frame}
}

{\fullbackground[scale=0.9,page=5]{ian-sensors.pdf}
    \begin{frame}{Quadrature output signal usage}
    \end{frame}
}

{\fullbackground[scale=0.9,page=6]{ian-sensors.pdf}
    \begin{frame}{Incremental encoder index signal}
    \end{frame}
}

{\fullbackground[scale=0.9,page=7]{ian-sensors.pdf}
    \begin{frame}{Using all edges increases resolution}
    \end{frame}
}

{\fullbackground[scale=0.9,page=8]{ian-sensors.pdf}
    \begin{frame}{Budget wheel encoders/sensors}
    \end{frame}
}

{\fullbackground[scale=0.9,page=9]{ian-sensors.pdf}
    \begin{frame}{Typical datasheet for encoder}
    \end{frame}
}

{\fullbackground[scale=0.9,page=10]{ian-sensors.pdf}
    \begin{frame}{Absolute optical encoders}
    \end{frame}
}

{\fullbackground[scale=0.9,page=11]{ian-sensors.pdf}
    \begin{frame}{Encoder signal conditioning}
    \end{frame}
}

{\fullbackground[scale=0.9,page=12]{ian-sensors.pdf}
    \begin{frame}{Interface to encoders}
    \end{frame}
}

{\fullbackground[scale=0.9,page=13]{ian-sensors.pdf}
    \begin{frame}{Interface to Arduino}
    \end{frame}
}

{\fullbackground[scale=0.9,page=14]{ian-sensors.pdf}
    \begin{frame}{Resolver position measurement}
    \end{frame}
}

{\fullbackground[scale=0.9,page=15]{ian-sensors.pdf}
    \begin{frame}{Resolver position measurement}
    \end{frame}
}

{\fullbackground[scale=0.9,page=16]{ian-sensors.pdf}
    \begin{frame}{Tachometer velocity measurement}
    \end{frame}
}

{\fullbackground[scale=0.9,page=17]{ian-sensors.pdf}
    \begin{frame}{Hall effect magnetic sensor}
    \end{frame}
}

{\fullbackground[scale=0.9,page=18]{ian-sensors.pdf}
    \begin{frame}{Hall effect magnetic sensor circuit}
    \end{frame}
}

{\fullbackground[scale=0.9,page=19]{ian-sensors.pdf}
    \begin{frame}{Hall effect magnetic rotary encoder}
    \end{frame}
}

{\fullbackground[scale=0.9,page=20]{ian-sensors.pdf}
    \begin{frame}{Magnetic rotary encoder}
    \end{frame}
}

{\fullbackground[scale=0.9,page=21]{ian-sensors.pdf}
    \begin{frame}{Electronic commutation systems}
    \end{frame}
}

{\fullbackground[scale=0.9,page=22]{ian-sensors.pdf}
    \begin{frame}{Block commutation}
    \end{frame}
}

{\fullbackground[scale=0.9,page=23]{ian-sensors.pdf}
    \begin{frame}{Components of an EC drive system}
    \end{frame}
}

\section{Measuring force}

{\fullbackground[scale=0.9,page=25]{ian-sensors.pdf}
    \begin{frame}{Definition of force}
    \end{frame}
}

{\fullbackground[scale=0.9,page=26]{ian-sensors.pdf}
    \begin{frame}{Four fundamental forces in nature}
    \end{frame}
}

{\fullbackground[scale=0.9,page=27]{ian-sensors.pdf}
    \begin{frame}{Forces arises from gravity acting on a mass}
    \end{frame}
}

{\fullbackground[scale=0.9,page=28]{ian-sensors.pdf}
    \begin{frame}{Measuring force by balancing known force}
    \end{frame}
}

{\fullbackground[scale=0.9,page=29]{ian-sensors.pdf}
    \begin{frame}{Measuring force by measuring strain}
    \end{frame}
}

{\fullbackground[scale=0.9,page=30]{ian-sensors.pdf}
    \begin{frame}{Piezo-electric effect}
    \end{frame}
}

{\fullbackground[scale=0.9,page=31]{ian-sensors.pdf}
    \begin{frame}{Measurement of elastic deformation}
    \end{frame}
}

{\fullbackground[scale=0.9,page=32]{ian-sensors.pdf}
    \begin{frame}{Direct measurement of elastic deformation}
    \end{frame}
}

{\fullbackground[scale=0.9,page=33]{ian-sensors.pdf}
    \begin{frame}{Strain gauge}
    \end{frame}
}

{\fullbackground[scale=0.9,page=34]{ian-sensors.pdf}
    \begin{frame}{Wheatstone bridge circuit}
    \end{frame}
}

{\fullbackground[scale=0.9,page=35]{ian-sensors.pdf}
    \begin{frame}{Wheatstone bridge circuit}
    \end{frame}
}

{\fullbackground[scale=0.9,page=36]{ian-sensors.pdf}
    \begin{frame}{Simple cantilever load cell}
    \end{frame}
}

{\fullbackground[scale=0.9,page=37]{ian-sensors.pdf}
    \begin{frame}{Measuring torque}
    \end{frame}
}

{\fullbackground[scale=0.9,page=38]{ian-sensors.pdf}
    \begin{frame}{Measuring torque using a gauge on shaft}
    \end{frame}
}

{\fullbackground[scale=0.9,page=39]{ian-sensors.pdf}
    \begin{frame}{1 - 6 DoF load cells}
    \end{frame}
}

{\fullbackground[scale=0.9,page=40]{ian-sensors.pdf}
    \begin{frame}{Inside a large 6DoF FT}
    \end{frame}
}

{\fullbackground[scale=0.9,page=41]{ian-sensors.pdf}
    \begin{frame}{Inside a small 6DoF FT}
    \end{frame}
}

{\fullbackground[scale=0.9,page=42]{ian-sensors.pdf}
    \begin{frame}{A signal conditioning amplifier is needed}
    \end{frame}
}


\begin{frame}{}
    \begin{center}
        \Large
        That's all, folks!\\[2em]
        \normalsize
        Questions:\\
        Portland Square B316 or \url{severin.lemaignan@plymouth.ac.uk} \\[1em]

        Slides:\\
        \href{https://github.com/severin-lemaignan/module-mobile-and-humanoid-robots}{\small
        github.com/severin-lemaignan/module-introduction-sensors-actuators}


    \end{center}
\end{frame}



\end{document}
