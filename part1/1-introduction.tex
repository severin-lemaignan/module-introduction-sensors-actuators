%!TEX program = xelatex

\documentclass[compress]{beamer}
%--------------------------------------------------------------------------
% Common packages
%--------------------------------------------------------------------------

\definecolor{links}{HTML}{663000}
\hypersetup{colorlinks,linkcolor=,urlcolor=links}

\usepackage[english]{babel}
\usepackage{pgfpages} % required for notes on second screen
\usepackage{graphicx}

\usepackage{multicol}


\usepackage{tabularx,ragged2e}
\usepackage{booktabs}

\setlength{\emergencystretch}{3em}  % prevent overfull lines
\providecommand{\tightlist}{%
  \setlength{\itemsep}{0pt}\setlength{\parskip}{0pt}}


\usetheme{hri}

\usepackage{remreset}% tiny package containing just the \@removefromreset command
\makeatletter
\@removefromreset{subsection}{section}
\makeatother
\setcounter{subsection}{1}

\newcommand{\source}[2]{{\tiny\it Source: \href{#1}{#2}}}

\usepackage{tikz}
\usetikzlibrary{mindmap,backgrounds,positioning}

\graphicspath{{figs/}}

\title{ROCO222 \newline Introduction to Sensors and Actuators}
\subtitle{Part 1 -- Introduction}
\date{}
\author{Séverin Lemaignan}
\institute{Centre for Neural Systems and Robotics\\{\bf Plymouth University}}

\begin{document}

\licenseframe{github.com/severin-lemaignan/module-introduction-sensors-actuators}

\maketitle

\begin{frame}{Who am I?}
    I am Dr. Séverin Lemaignan. My office is B316 Portland Square
\end{frame}

\begin{frame}{And what do I do?}

\begin{multicols}{2}

    \textbf{Cognitive robotics}

  Building robots and their artificial intelligence inspired on natural
  systems, such as developing children

    \textbf{Human-Robot Interaction}

  Building robots that can work alongside people, using social cues that
  people use to communicate

    \begin{center}
        \includegraphics[width=0.8\linewidth]{cowriter}
    \end{center}

\end{multicols}
\end{frame}

\begin{frame}{Availability}

    \begin{itemize}
        \item I have lots of time (2 hours) to deal with your queries in the laboratory sessions
        \item  Otherwise, the best thing to do is to email me first and see if I can sort out your query. If not then I will arrange a time to meet up with you.
        \item Don't forget – if you have any problems, or even if you are not sure about something - just email me

    \end{itemize}

    \begin{center}
        severin.lemaignan@plymouth.ac.uk
    \end{center}
\end{frame}

\begin{frame}{Registration}
        \centering
        \includegraphics[width=0.4\linewidth]{registration1}
        \includegraphics[width=0.4\linewidth]{registration2}
\end{frame}

\begin{frame}{Special needs}

    \begin{itemize}

    \item Anyone with dyslexia or any other special needs that requires any form
        of support during examinations and tests should email me immediately.

    \item Please do not ASSUME that I know of your condition. It takes the
        University some weeks to process data and forward it to the correct
            members of staff.

    \item Even if you are currently undergoing assessment, it is advisable to
        contact me and I will organize suitable support for you during your
            tests and exams.

    \item Contact me if any issues relating to harassment, inclusivity or if
        language/cultural difficulties arise.

    \end{itemize}
\end{frame}

\begin{frame}{Labs}
    \begin{itemize}
        \item ROCO222 is also taught and supervised in a computer laboratory for
            a TWO HOURS EACH WEEK.
        \item There are 2 labs so please go to your timetabled group and pair up
            with someone
    \end{itemize}

    \centering
          \begin{tabular}{@{}llll@{}}
                \toprule
                Day?                & Time? & Where?   & Who? \\ \midrule
                Monday -- Lab 1/\textbf{02} & 09:00--11:00     & SMB 307  & Jake \& myself            \\
                Thursday -- Lab 1/\textbf{01} & 11:00--13:00     & SMB 307  & Jake \& myself          \\ \bottomrule
            \end{tabular}


\end{frame}

\section{Assessment}

\begin{frame}{Assessment}
    \begin{exampleblock}{Formative vs Summative}
        \textbf{Formative}: during the term; \textbf{Summative}: at the end of
        the term.
    \end{exampleblock}

    \begin{itemize}
        \item There will be practical assessments in the form of laboratory
            practical work (60\%)
        \item First practical needs to be submitted by 16:00, 13th October 2017
        \item Feedback within 20 working days
        \item Complete lab journal submitted Thursday 16:00, 11th January 2018
        \item There will be a final written exam (40\%) -- \textbf{note that the
            exam questions may be on material discussed during the labs}
    \end{itemize}

\end{frame}


\section{Robotics: what are your expectations?}

\begin{frame}[plain]
    \begin{center}
        \Large
    \href{https://www.mentimeter.com/s/d42456cc370aa465642c7e815a68646b/4fdc61757b92}{Go to www.menti.com and use the code 60 34 45}
    \end{center}
\end{frame}



\section{Module Overview}

\videoframe[0.56]{part1/figs/NISSAN-ProPILOT-chair.mp4}


\begin{frame}{Module aims}
    \begin{itemize}
        \item<+-> Develop an in-depth understanding of what electical motors are,
            how they work (hands-on!), how they are characterised
        \item<+-> Learn how to control them, and program an embedded controller
            for your own motor
        \item<+-> Get a first hand-on experience with a complete robotic system:
            from the hardware, to the low-level closed-loop control, to
            system-level visualisation
        \item<+-> We have a bit of room to touch one or two other topics
    \end{itemize}
\end{frame}

\begin{frame}{Learning outcomes}
    \begin{enumerate}
        \item<+-> Demonstrate knowledge and critical understanding of the operating
            \textbf{characteristics of electrical motors} (brushed, brushless,
            servo, stepper motors...)
        \item<+-> Explore (some) robot sensors, integrate them into software
        \item<+-> Comprehend and characterise the effects of \textbf{closing the speed and current
            loops} in drive systems; demonstrate it practically by \textbf{creating a
            closed loop motor system}
        \item<+-> Understand the fundamentals of the \textbf{ROS middleware},
            and use it to control motors and run simple robot visualisations
        \item<+-> Feel confident when using the \textbf{Linux command-line}; know how to use and reflect on \textbf{document/code versioning and
            sharing} (with git)
    \end{enumerate}
\end{frame}

\begin{frame}{In your curriculum...}
    \begin{itemize}
        \item Last year: quite a lot of electronics, \textbf{ROCO103PP}?
        \item Next term: \textbf{ROCO224} (Martin Stoelen) -- kinematics, mechanical engineering,
            robot control
        \item Next year: \textbf{ROCO318} -- sensors, algorithms, AI (path
            planning, localisation)
    \end{itemize}
\end{frame}


\begin{frame}{This year: ROCO222 and ROCO224}

    \only<1>{
    \Large
    \textbf{From...}

    \begin{center}
        \includegraphics[width=0.5\paperheight]{better-dc-motor}
    \end{center}

    }
    \only<2> {
        \Large
    \textbf{...to}

    \begin{center}
        \includegraphics[width=0.6\paperheight]{servo-arm}
    \end{center}

    }

\end{frame}

\begin{frame}{This module: syllabus}

\begin{itemize}
    \item<+-> \textbf{Week 1} -- Arduino + Encoders
    \item<+-> \textbf{Week 2} -- The physics behind motors: electromagnetism, induction, magnetic
        force
    \item<+-> \textbf{Week 4} -- DC motors, brushed \& brushless
    \item<+-> \textbf{Week 5} -- Closed-loop motor control, PWM, PID, H-bridge
    \item<+-> \textbf{Week 6} -- Induction motors
    \item<+-> \textbf{Week 7} -- Servo-motors \& stepper motors
    \item<+-> \textbf{Week 9} -- Software engineering for robotics
    \item<+-> \textbf{Week 10} -- ROS middleware, joint state, kinematics 101, visualisation
    \item<+-> \textbf{Week 11} -- Sensors
\end{itemize}

\end{frame}

\section{Laboratory Coursework}

\begin{frame}{This week: Lab journal \& Command-line 101}
    \begin{itemize}
        \item Document versioning: what is GIT?
        \item Creating \& managing a lab journal with GIT and markdown
        \item Linux command-line
        \item Remotely connecting to a robot
    \end{itemize}

    \begin{exampleblock}{Tomorrow's practical}
        If you want to, you can come from 11:00 to 13:00 to finish Monday's
        practical (including connecting to the robot)
    \end{exampleblock}



\end{frame}

\begin{frame}{Build a DC motor}
    \begin{columns}
        \begin{column}{0.45\linewidth}
            Build a DC brushed motor from first principles
            \begin{itemize}

                \item Wind armature

                \item Position field magnets

                \item Build commutator

                \item Test operation

                \item Measure characteristics

            \end{itemize}

        \end{column}
        \begin{column}{0.45\linewidth}

            \begin{center}
                \includegraphics[width=0.6\columnwidth]{better-dc-motor}\\
                \includegraphics[width=0.8\columnwidth]{motor-characterisation}
            \end{center}

        \end{column}
    \end{columns}
\end{frame}

\begin{frame}{Introduction to Arduino and simple motor control}
    \begin{columns}
        \begin{column}{0.45\linewidth}

            \begin{itemize}
                \item Controlling your DC motor using an Arduino
                \item Build a speed-controlled electric fan \eg use
                    temperature sensor

                \item Control an RC servo using an Arduino
            \end{itemize}
        \end{column}
        \begin{column}{0.45\linewidth}

            \begin{center}
                \includegraphics[width=0.8\columnwidth]{arduino-uno}\\
                \includegraphics[width=0.8\columnwidth]{dc-motor-fan}\\
                \includegraphics[width=0.8\columnwidth]{servo}\\
            \end{center}

        \end{column}
    \end{columns}
\end{frame}


\begin{frame}{Build an incremental encoder}
    \begin{columns}
        \begin{column}{0.45\linewidth}

            \begin{itemize}
                    \item Build a simple encoder from first principles
                    \item Use a LED and phototransistor
                    \item Measure rotational speed of a small DC motor
                    \item Integrate it with the Arduino
            \end{itemize}
        \end{column}
        \begin{column}{0.45\linewidth}

            \begin{center}
                \includegraphics[width=0.9\columnwidth]{encoder}\\
            \end{center}

        \end{column}
    \end{columns}
\end{frame}


\begin{frame}{Write a PID controller for your DC motor}
    \begin{center}
        \includegraphics[width=0.8\linewidth]{pid}
    \end{center}
\end{frame}

\begin{frame}{Control a stepper motor with an Arduino}
    \begin{center}
        \includegraphics[width=0.8\linewidth]{stepper}
    \end{center}

    \begin{itemize}
        \item Control a stepper motor from the Arduino
        \item Implement stepper modes from first principles
    \end{itemize}
\end{frame}

\begin{frame}{Mini project: a 3DoF robot arm controlled from ROS}
    \begin{center}
        \includegraphics[width=0.8\linewidth]{arm-rviz}
    \end{center}

    3 weeks to:

    \begin{itemize}
        \item Design and build a 3DoF (one stepper, 2 servos) robot arm
        \item Create a 3D model and visualise it
        \item Control it with ROS
    \end{itemize}

\end{frame}

\begin{frame}{Attention: I am away on the 11th of October}
    \begin{itemize}
        \item On 11th October: no lecture
        \item On 12th October: practical with Jake
    \end{itemize}
\end{frame}

%%%%%%%%%%%%%%%%%%%%%%%%%%%%%%%%%%%%%%%%%%%%%%%%%%%%%%%%%%%%%%%%%%%%%%%

\begin{frame}[plain]
    \begin{center}
        \Large
        10 min break, and off we start!\\[2em]
    \end{center}
\end{frame}

%%%%%%%%%%%%%%%%%%%%%%%%%%%%%%%%%%%%%%%%%%%%%%%%%%%%%%%%%%%%%%%%%%%%%%%
\section{Intro to Arduino}


\begin{frame}{The Arduino microcontroller}
    \begin{columns}
        \begin{column}{0.5\linewidth}
            \begin{itemize}
                \item What is an Arduino board?
                \item Developing programs
                \item Inputs and outputs
                \item Running control loops
                \item Using PWM to control motors
            \end{itemize}
        \end{column}
        \begin{column}{0.5\linewidth}
            \begin{center}
                \includegraphics[width=0.8\linewidth]{arduino-uno}
            \end{center}
        \end{column}
    \end{columns}

    \pause

    \begin{itemize}
        \item Arduino is an open-source electronics platform based on easy to
            use hardware and software
        \item it is intended for anyone making interactive projects
        \item very relevant to robotics, including professional-grade projects!
    \end{itemize}
\end{frame}


\begin{frame}{The Arduino microcontroller ecosystem}
    \begin{columns}
        \begin{column}{0.3\linewidth}
            \begin{center}
                \includegraphics[width=\linewidth]{arduino-uno}
                Controller
            \end{center}
        \end{column}
        \begin{column}{0.3\linewidth}
            \begin{center}
                \includegraphics[width=\linewidth]{wifi-shield}
                Wifi shield
            \end{center}

        \end{column}
        \begin{column}{0.3\linewidth}
            \begin{center}
            \includegraphics[width=\linewidth]{motor-shield}
            Motor shield
            \end{center}

        \end{column}
    \end{columns}

    \vspace{0.5em}

    \begin{columns}
        \begin{column}{0.25\linewidth}
            \begin{center}
                \includegraphics[width=0.9\linewidth]{us-sensor}
                Ultrasonic sensor
            \end{center}
        \end{column}
        \begin{column}{0.25\linewidth}
            \begin{center}
                \includegraphics[width=0.9\linewidth]{robot-car}
                Robot car
            \end{center}
        \end{column}
         \begin{column}{0.5\linewidth}
             "Arduino is an open-source electronics prototyping
             platform based on flexible,
             easy-to-use hardware and
             software. It's intended for
             artists, designers, hobbyists
             and anyone interested in
             creating interactive objects
             or environments"
        \end{column}
    \end{columns}
\end{frame}

\imageframe[color=white, scale=0.95]{arduino-boards}

\begin{frame}{Arduino Uno}
    \begin{columns}
        \begin{column}{0.5\linewidth}
            \begin{itemize}
                \item Relatively robust board
                \item Microcontroller board based on the ATmega328
                \item 14 digital Input/Output pins
                \item 6 outputs can be used as PWM outputs
                \item 6 analog inputs
                \item 16MHz ceramic resonator
                \item USB connection
                \item power jack
                \item ICSP header
                \item reset button
            \end{itemize}

        \end{column}
        \begin{column}{0.5\linewidth}
            \begin{center}
                \includegraphics[width=\linewidth]{arduino-uno}
            \end{center}
        \end{column}
    \end{columns}
\end{frame}

{\fullbackground[scale=0.9,page=6]{ian-intro-arduino.pdf}
    \begin{frame}{Arduino Uno}
    \end{frame}
}

\imageframe{arduino-schematic}

{\fullbackground[scale=0.9,page=8]{ian-intro-arduino.pdf}
    \begin{frame}{Arduino Uno summary}
    \end{frame}
}

    \begin{frame}{Powering Arduino Uno}
        \begin{itemize}
            \item via the USB connection
            \item external power supply
            \item the power source is selected automatically
            \item external (non-USB) 2.1mm centre-positive plug into the board's
                power jack
            \item leads from a battery can be inserted in the \texttt{Gnd} and
                \texttt{Vin} pin
                headers of the \texttt{POWER} connector
            \item recommended range is 7 to 12 volts
        \end{itemize}
    \end{frame}

\begin{frame}{Some pins have specialized functions}

    \only<1> {
    \begin{itemize}
        \item Serial: 0 (\texttt{RX}) and 1 (\texttt{TX}) \\
                Used to receive (\texttt{RX}) and transmit (\texttt{TX}) TTL
                serial data
        \item external interrupts: 2 and 3 \\
                Can be configured to trigger an interrupt on a low value, a
                rising or falling edge, or a change in value
        \item PWM: 3, 5, 6, 9, 10 and 11 \\
                Provide 8-bit PWM output with the \cpp{analogWrite()} function
    \end{itemize}
}
    \only<2> {
        \begin{itemize}
            \item SPI: 10 (\texttt{SS}), 11 (\texttt{MOSI}), 12 (\texttt{MISO}),
                13 (\texttt{SCK}) \\
                 Support SPI communication using the SPI library
            \item LED: 13. Built-in LED connected to digital pin 13
            \item The Uno has 6 analog inputs, labeled \texttt{A0} through
                \texttt{A5}
            \item I$^2$C: \texttt{A4} or \texttt{SDA} pin and \texttt{A5} or
                \texttt{SCL} pin \\
                  Support I$^2$C communication using the \texttt{Wire} library
        \end{itemize}
    }
\end{frame}

    \begin{frame}{How to start coding: software setup}
        \begin{columns}
            \begin{column}{0.5\linewidth}
                \begin{center}
                    \includegraphics[width=0.9\linewidth]{arduino-ide}
                \end{center}
            \end{column}
            \begin{column}{0.5\linewidth}
                \begin{itemize}
                    \item install the IDE (download from \url{www.arduino.cc} or
                        \sh{apt install arduino})
                    \item open the Arduino IDE
                    \item use a USB A to B cable to plug the board into your
                        computer
                    \item setup the board type and serial port from the
                        \emph{Tools} menu
                \end{itemize}
            \end{column}
        \end{columns}
    \end{frame}

    \begin{frame}{Programming the Arduino}
        \begin{itemize}
            \item Arduino's programming language is a version of C
            \item Arduino calls a program a \textbf{sketch}. Sketches are saved
                by default in your \textbf{sketchbook} (cf \emph{File >
                Preferences})
            \item Most C commands work in the sketch editor
            \item 4 basic compoenents of an Arduino program:
                \begin{itemize}
                    \item initialization
                    \item setup
                    \item loop
                    \item user defined functions
                \end{itemize}
            \item Arduino Uno is designed to allow it to be remotely reset by
                software

        \end{itemize}
    \end{frame}

{\fullbackground[scale=0.9,page=14]{ian-intro-arduino.pdf}
    \begin{frame}{Programming the Arduino: initialization}
    \end{frame}
}

\begin{frame}[fragile]{Programming the Arduino: setup}
    \begin{itemize}
        \item Initialization for the Arduino board
        \item Usually initiate Serial communication
        \item Usually assign digital pins to input or output
        \item For example:
    \end{itemize}

            \begin{cppcode}
void setup() // 'void' means setup() does not return anything
{
    pinMode(13, OUPUT); // pinMode sets the state of a digital pin,
    pinMode(12, INPUT); // either to OUPUT or to INPUT

    Serial.begin(9600); // initialises a serial console at 9600 bauds
}

            \end{cppcode}
\end{frame}


\begin{frame}[fragile]{Programming the Arduino: main loop}
    \begin{itemize}
        \item The loop is the main portion of your program
        \item The microcontroller runs it forever
        \item For instance:
    \end{itemize}

            \begin{cppcode}
void loop()
{
    digitalWrite(13, HIGH); // set the LED on
    delay(1000);            // wait for one second
    digitalWrite(13, LOW);  // set the LED off
    delay(1000);            // wait for one second
}

            \end{cppcode}
\end{frame}


{\fullbackground[scale=0.9,page=17]{ian-intro-arduino.pdf}
    \begin{frame}{Programming the Arduino: user defined functions}
    \end{frame}
}

{\fullbackground[scale=0.9,page=18]{ian-intro-arduino.pdf}
    \begin{frame}{Programming the Arduino: user defined functions}
    \end{frame}
}

{\fullbackground[scale=0.9,page=19]{ian-intro-arduino.pdf}
    \begin{frame}{Programming the Arduino: digital I/O}
    \end{frame}
}

{\fullbackground[scale=0.9,page=20]{ian-intro-arduino.pdf}
    \begin{frame}{Programming the Arduino: analog I/O}
    \end{frame}
}

{\fullbackground[scale=0.9,page=21]{ian-intro-arduino.pdf}
    \begin{frame}{Programming the Arduino: serial communications}
    \end{frame}
}

{\fullbackground[scale=0.9,page=22]{ian-intro-arduino.pdf}
    \begin{frame}{Programming the Arduino: serial library}
    \end{frame}
}

{\fullbackground[scale=0.9,page=23]{ian-intro-arduino.pdf}
    \begin{frame}{Programming the Arduino: simple control}
    \end{frame}
}

{\fullbackground[scale=0.9,page=24]{ian-intro-arduino.pdf}
    \begin{frame}{Compiling and uploading code}
    \end{frame}
}

\section{Program Examples}

\begin{frame}[fragile]{LED blinking}
    \begin{columns}
        \begin{column}{0.7\linewidth}
\begin{cppcode}
// Pin 13 has an LED connected on most Arduino boards.
// give it a name:
int led = 13;

// the setup routine runs once when you press reset:
void setup() {                
  // initialize the digital pin as an output.
  pinMode(led, OUTPUT);     
}

// the loop routine runs over and over again forever:
void loop() {
  digitalWrite(led, HIGH);   // turn the LED on (HIGH is the voltage level)
  delay(1000);               // wait for a second
  digitalWrite(led, LOW);    // turn the LED off by making the voltage LOW
  delay(1000);               // wait for a second
}
\end{cppcode}

        \end{column}
        \begin{column}{0.3\linewidth}
            \begin{center}
                \includegraphics[width=0.35\paperheight]{arduino-led}
                \includegraphics[width=0.35\paperheight]{arduino-led-schematic}
            \end{center}
        \end{column}
    \end{columns}
\end{frame}


\begin{frame}[fragile]{Encoder}
    \begin{columns}
        \begin{column}{0.7\linewidth}
\begin{cppcode}
#include <Encoder.h>

// Change these two numbers to the pins connected to your encoder.
Encoder myEnc(5, 6);
//   avoid using pins with LEDs attached

void setup() {
  Serial.begin(9600);
  Serial.println("Basic Encoder Test:");
}

long oldPosition  = -999;

void loop() {
  long newPosition = myEnc.read();
  if (newPosition != oldPosition) {
    oldPosition = newPosition;
    Serial.println(newPosition);
  }
}
\end{cppcode}

        \end{column}
        \begin{column}{0.3\linewidth}
            \begin{center}
                \includegraphics[width=\columnwidth]{encoders3}
            \end{center}
        \end{column}
    \end{columns}
\end{frame}






%%%%%%%%%%%%%%%%%%%%%%%%%%%%%%%%%%%%%%%%%%%%%%%%%%%%%%%%%%%%%%%%%%%%%%%
\section{Measuring position: encoders}

{\fullbackground[scale=0.9,page=2]{ian-sensors.pdf}
    \begin{frame}{Digital incremental encoder}
    \end{frame}
}
{\fullbackground[scale=0.9,page=3]{ian-sensors.pdf}
    \begin{frame}{Optical incremental encoder}
    \end{frame}
}

{\fullbackground[scale=0.9,page=4]{ian-sensors.pdf}
    \begin{frame}{Quadrature output signal generation}
    \end{frame}
}

{\fullbackground[scale=0.9,page=5]{ian-sensors.pdf}
    \begin{frame}{Quadrature output signal usage}
    \end{frame}
}

{\fullbackground[scale=0.9,page=6]{ian-sensors.pdf}
    \begin{frame}{Incremental encoder index signal}
    \end{frame}
}

{\fullbackground[scale=0.9,page=7]{ian-sensors.pdf}
    \begin{frame}{Using all edges increases resolution}
    \end{frame}
}

{\fullbackground[scale=0.9,page=8]{ian-sensors.pdf}
    \begin{frame}{Budget wheel encoders/sensors}
    \end{frame}
}

{\fullbackground[scale=0.9,page=9]{ian-sensors.pdf}
    \begin{frame}{Typical datasheet for encoder}
    \end{frame}
}

{\fullbackground[scale=0.9,page=10]{ian-sensors.pdf}
    \begin{frame}{Absolute optical encoders}
    \end{frame}
}

{\fullbackground[scale=0.9,page=11]{ian-sensors.pdf}
    \begin{frame}{Encoder signal conditioning}
    \end{frame}
}

{\fullbackground[scale=0.9,page=12]{ian-sensors.pdf}
    \begin{frame}{Interface to encoders}
    \end{frame}
}

{\fullbackground[scale=0.9,page=13]{ian-sensors.pdf}
    \begin{frame}{Interface to Arduino}
    \end{frame}
}

{\fullbackground[scale=0.9,page=14]{ian-sensors.pdf}
    \begin{frame}{Resolver position measurement}
    \end{frame}
}

{\fullbackground[scale=0.9,page=15]{ian-sensors.pdf}
    \begin{frame}{Resolver position measurement}
    \end{frame}
}

{\fullbackground[scale=0.9,page=16]{ian-sensors.pdf}
    \begin{frame}{Tachometer velocity measurement}
    \end{frame}
}

{\fullbackground[scale=0.9,page=17]{ian-sensors.pdf}
    \begin{frame}{Hall effect magnetic sensor}
    \end{frame}
}

{\fullbackground[scale=0.9,page=18]{ian-sensors.pdf}
    \begin{frame}{Hall effect magnetic sensor circuit}
    \end{frame}
}

{\fullbackground[scale=0.9,page=19]{ian-sensors.pdf}
    \begin{frame}{Hall effect magnetic rotary encoder}
    \end{frame}
}

{\fullbackground[scale=0.9,page=20]{ian-sensors.pdf}
    \begin{frame}{Magnetic rotary encoder}
    \end{frame}
}

{\fullbackground[scale=0.9,page=21]{ian-sensors.pdf}
    \begin{frame}{Electronic commutation systems}
    \end{frame}
}

{\fullbackground[scale=0.9,page=22]{ian-sensors.pdf}
    \begin{frame}{Block commutation}
    \end{frame}
}

{\fullbackground[scale=0.9,page=23]{ian-sensors.pdf}
    \begin{frame}{Components of an EC drive system}
    \end{frame}
}


\end{document}
