%!TEX program = xelatex

\documentclass[compress]{beamer}
%--------------------------------------------------------------------------
% Common packages
%--------------------------------------------------------------------------

\definecolor{links}{HTML}{663000}
\hypersetup{colorlinks,linkcolor=,urlcolor=links}

\usepackage[english]{babel}
\usepackage{pgfpages} % required for notes on second screen
\usepackage{graphicx}

\usepackage{multicol}

\usepackage{tabularx,ragged2e}
\usepackage{booktabs}

\setlength{\emergencystretch}{3em}  % prevent overfull lines
\providecommand{\tightlist}{%
  \setlength{\itemsep}{0pt}\setlength{\parskip}{0pt}}


\usetheme{hri}

% Display the navigation bullet even without subsections
\usepackage{remreset}% tiny package containing just the \@removefromreset command
\makeatletter
\@removefromreset{subsection}{section}
\makeatother
\setcounter{subsection}{1}


\newcommand{\source}[2]{{\tiny\it Source: \href{#1}{#2}}}

\usepackage{tikz}
\usetikzlibrary{mindmap,backgrounds,positioning,calc,patterns}
\usepackage{pgfplots}
\pgfplotsset{compat=newest}
\usepackage{circuitikz}

\graphicspath{{part5/figs/}}

\title{ROCO222 \newline Intro to Sensors and Actuators}
\subtitle{Part 5 -- Motor control}

\date{}
\author{Séverin Lemaignan}
\institute{Centre for Neural Systems and Robotics\\{\bf Plymouth University}}

\begin{document}

\licenseframe{github.com/severin-lemaignan/module-mobile-and-humanoid-robots}

\maketitle

\section{Simple motor control}

{\fullbackground[scale=0.9,page=2]{ian-simple-motor-ctrl.pdf}
    \begin{frame}{Open loop control}
    \end{frame}
}
{\fullbackground[scale=0.9,page=3]{ian-simple-motor-ctrl.pdf}
    \begin{frame}{Closed loop control}
    \end{frame}
}

{\fullbackground[scale=0.9,page=4]{ian-simple-motor-ctrl.pdf}
    \begin{frame}{Motor operating point voltage dependent}
    \end{frame}
}

{\fullbackground[scale=0.9,page=5]{ian-simple-motor-ctrl.pdf}
    \begin{frame}{Simple motor speed control}
    \end{frame}
}

{\fullbackground[scale=0.9,page=6]{ian-simple-motor-ctrl.pdf}
    \begin{frame}{Linear power stage}
    \end{frame}
}

{\fullbackground[scale=0.9,page=7]{ian-simple-motor-ctrl.pdf}
    \begin{frame}{Better motor speed control}
    \end{frame}
}

{\fullbackground[scale=0.9,page=8]{ian-simple-motor-ctrl.pdf}
    \begin{frame}{Pulsed power stage (PWM)}
    \end{frame}
}

{\fullbackground[scale=0.9,page=9]{ian-simple-motor-ctrl.pdf}
    \begin{frame}{Pulse width modulation waveforms}
    \end{frame}
}

{\fullbackground[scale=0.9,page=10]{ian-simple-motor-ctrl.pdf}
    \begin{frame}{PWM and current ripple}
    \end{frame}
}

{\fullbackground[scale=0.9,page=11]{ian-simple-motor-ctrl.pdf}
    \begin{frame}{PWM and electronic commutation}
    \end{frame}
}

\section{H-bridge}

{\fullbackground[scale=0.9,page=02]{ian-hbridge.pdf}
    \begin{frame}{H-bridge}
    \end{frame}
}


{\fullbackground[scale=0.9,page=03]{ian-hbridge.pdf}
    \begin{frame}{H-bridge control motor direction}
    \end{frame}
}


{\fullbackground[scale=0.9,page=04]{ian-hbridge.pdf}
    \begin{frame}{Using junction transistors as switches}
    \end{frame}
}


{\fullbackground[scale=0.9,page=05]{ian-hbridge.pdf}
    \begin{frame}{Motor inductance}
    \end{frame}
}


{\fullbackground[scale=0.9,page=06]{ian-hbridge.pdf}
    \begin{frame}{Back EMF chen current abruptly switched off}
    \end{frame}
}


{\fullbackground[scale=0.9,page=07]{ian-hbridge.pdf}
    \begin{frame}{Diode protection against back EMF}
    \end{frame}
}


{\fullbackground[scale=0.9,page=08]{ian-hbridge.pdf}
    \begin{frame}{Simple transistor H-Bridge}
    \end{frame}
}


{\fullbackground[scale=0.9,page=09]{ian-hbridge.pdf}
    \begin{frame}{H-Bridges are available off-the-shelf}
    \end{frame}
}


\section{Motor control with Arduino}

{\fullbackground[scale=0.9,page=02]{ian-arduino-dc-motors.pdf}
    \begin{frame}{Arduino motor shield}
    \end{frame}
}

{\fullbackground[scale=0.9,page=03]{ian-arduino-dc-motors.pdf}
    \begin{frame}{Arduino motor shield specs}
    \end{frame}
}

{\fullbackground[scale=0.9,page=04]{ian-arduino-dc-motors.pdf}
    \begin{frame}{L298 dual full-bridge driver}
    \end{frame}
}

{\fullbackground[scale=0.9,page=05]{ian-arduino-dc-motors.pdf}
    \begin{frame}{L298 dual full-bridge driver}
    \end{frame}
}

{\fullbackground[scale=0.9,page=06]{ian-arduino-dc-motors.pdf}
    \begin{frame}{Motor shield schematic}
    \end{frame}
}

{\fullbackground[scale=0.9,page=07]{ian-arduino-dc-motors.pdf}
    \begin{frame}{Install the Arduino motor shield}
    \end{frame}
}

{\fullbackground[scale=0.9,page=08]{ian-arduino-dc-motors.pdf}
    \begin{frame}{Arduino motor shield power supply}
    \end{frame}
}

{\fullbackground[scale=0.9,page=09]{ian-arduino-dc-motors.pdf}
    \begin{frame}{Arduino motor shield output channels}
    \end{frame}
}

{\fullbackground[scale=0.9,page=10]{ian-arduino-dc-motors.pdf}
    \begin{frame}{Pins always in use by the motor shield}
    \end{frame}
}

{\fullbackground[scale=0.9,page=11]{ian-arduino-dc-motors.pdf}
    \begin{frame}{Motor shield pin usage}
    \end{frame}
}

{\fullbackground[scale=0.9,page=13]{ian-arduino-dc-motors.pdf}
    \begin{frame}{DC motor connections}
    \end{frame}
}

{\fullbackground[scale=0.9,page=14]{ian-arduino-dc-motors.pdf}
    \begin{frame}{Motor shield 1-channel DC motor demo}
    \end{frame}
}

\begin{frame}[fragile]{Motor shield 1-channel DC motor demo}

\begin{cppcode}
/*****************************************************
Motor Shield 1-Channel DC Motor Demo
by Randy Sarafan
For more information see:
www.instructables.com/id/Arduino-Motor-Shield-Tutorial
*****************************************************/
void setup() {
//Setup Channel A
pinMode(12, OUTPUT); //Initiates Motor Channel A pin
pinMode(9, OUTPUT); //Initiates Brake Channel A pin
}
\end{cppcode}

\end{frame}

\begin{frame}[fragile]{Motor shield 1-channel DC motor demo}

\begin{cppcode}
void loop(){
  //forward @ full speed
  digitalWrite(12, HIGH); //Establishes forward direction of Channel A
  digitalWrite(9, LOW); //Disengage the Brake for Channel A
  analogWrite(3, 255); //Spins the motor on Channel A at full speed
  delay(3000);
  digitalWrite(9, HIGH); //Engage the Brake for Channel A
  delay(1000);
  //backward @ half speed
  digitalWrite(12, LOW); //Establishes backward direction of Channel A
  digitalWrite(9, LOW); //Disengage the Brake for Channel A
  analogWrite(3, 123); //Spins the motor on Channel A at half speed
  delay(3000);
  digitalWrite(9, HIGH); //Engage the Brake for Channel A
  delay(1000);
}
\end{cppcode}

\end{frame}


{\fullbackground[scale=0.9,page=17]{ian-arduino-dc-motors.pdf}
    \begin{frame}{DC motor running}
    \end{frame}
}

{\fullbackground[scale=0.9,page=18]{ian-arduino-dc-motors.pdf}
    \begin{frame}{Motor shield 2-channels DC motor demo}
    \end{frame}
}

\begin{frame}[fragile]{Motor shield 2-channels DC motor demo}

\begin{cppcode}
/*************************************************************
Motor Shield 2‐Channel DC Motor Demo
by Randy Sarafan
  
For more information see:
www.instructables.com/id/Arduino-Motor‐Shield‐Tutorial
*************************************************************/
  
  
void  setup()
{
  //Setup Channel A
  pinMode(12, OUTPUT);  //Initiates  Motor  Channel  A  pin
  pinMode(9,  OUTPUT);  //Initiates  Brake  Channel  A  pin
  
  //Setup  Channel  B
  pinMode(13,  OUTPUT);  //Initiates  Motor  Channel  A  pin
  pinMode(8,  OUTPUT);    //Initiates  Brake  Channel  A  pin
}
\end{cppcode}

\end{frame}

\begin{frame}[fragile]{Motor shield 2-channels DC motor demo}

\begin{cppcode}
void loop(){
  //Motor A forward @ full speed
  digitalWrite(12, HIGH); //Establishes forward direction of Channel A
  digitalWrite(9, LOW); //Disengage the Brake for Channel A
  analogWrite(3, 255); //Spins the motor on Channel A at full speed

  //Motor B backward @ half speed
  digitalWrite(13, LOW); //Establishes backward direction of Channel B
  digitalWrite(8, LOW); //Disengage the Brake for Channel B
  analogWrite(11, 123); //Spins the motor on Channel B at half speed
  delay(3000);

  digitalWrite(9, HIGH); //Engage the Brake for Channel A
  digitalWrite(9, HIGH); //Engage the Brake for Channel B
  delay(1000);

  //Motor A forward @ full speed
  digitalWrite(12, LOW); //Establishes backward direction of Channel A
  digitalWrite(9, LOW); //Disengage the Brake for Channel A
  analogWrite(3, 123); //Spins the motor on Channel A at half speed

  //Motor B forward @ full speed
  digitalWrite(13, HIGH); //Establishes forward direction of Channel B
  digitalWrite(8, LOW); //Disengage the Brake for Channel B
  analogWrite(11, 255); //Spins the motor on Channel B at full speed
  delay(3000);

  digitalWrite(9, HIGH); //Engage the Brake for Channel A
  digitalWrite(9, HIGH); //Engage the Brake for Channel B
  delay(1000);
}
\end{cppcode}

\end{frame}


\section{Four quadrant control}

{\fullbackground[scale=0.9,page=02]{ian-motor-ctrl.pdf}
    \begin{frame}{Motor speed-torque chart}
    \end{frame}
}

{\fullbackground[scale=0.9,page=03]{ian-motor-ctrl.pdf}
    \begin{frame}{One and two quadrant operation}
    \end{frame}
}

{\fullbackground[scale=0.9,page=04]{ian-motor-ctrl.pdf}
    \begin{frame}{Four quadrant operation -- hoist example}
    \end{frame}
}

{\fullbackground[scale=0.9,page=05]{ian-motor-ctrl.pdf}
    \begin{frame}{Four quadrant operation}
    \end{frame}
}

\section{Feedback control}

{\fullbackground[scale=0.9,page=07]{ian-motor-ctrl.pdf}
    \begin{frame}{Open loop control systems}
    \end{frame}
}

{\fullbackground[scale=0.9,page=08]{ian-motor-ctrl.pdf}
    \begin{frame}{Open loop control systems}
    \end{frame}
}

{\fullbackground[scale=0.9,page=09]{ian-motor-ctrl.pdf}
    \begin{frame}{Simple feedback controller}
    \end{frame}
}

{\fullbackground[scale=0.9,page=10]{ian-motor-ctrl.pdf}
    \begin{frame}{Simple feedback controller}
    \end{frame}
}

{\fullbackground[scale=0.9,page=11]{ian-motor-ctrl.pdf}
    \begin{frame}{Simple feedback controller}
    \end{frame}
}

{\fullbackground[scale=0.9,page=12]{ian-motor-ctrl.pdf}
    \begin{frame}{Adding a PID controller}
    \end{frame}
}

{\fullbackground[scale=0.9,page=13]{ian-motor-ctrl.pdf}
    \begin{frame}{PID parallel pathways}
    \end{frame}
}

{\fullbackground[scale=0.9,page=14]{ian-motor-ctrl.pdf}
    \begin{frame}{Proportional term}
    \end{frame}
}

{\fullbackground[scale=0.9,page=15]{ian-motor-ctrl.pdf}
    \begin{frame}{Integral term}
    \end{frame}
}

{\fullbackground[scale=0.9,page=16]{ian-motor-ctrl.pdf}
    \begin{frame}{Derivative term}
    \end{frame}
}

{\fullbackground[scale=0.9,page=17]{ian-motor-ctrl.pdf}
    \begin{frame}{Changing PID controller characteristics}
    \end{frame}
}

{\fullbackground[scale=0.9,page=18]{ian-motor-ctrl.pdf}
    \begin{frame}{PID differential equation}
    \end{frame}
}

{\fullbackground[scale=0.9,page=19]{ian-motor-ctrl.pdf}
    \begin{frame}{Transfer function of PID controller}
    \end{frame}
}

{\fullbackground[scale=0.9,page=20]{ian-motor-ctrl.pdf}
    \begin{frame}{Example: overall open-loop transfer function}
    \end{frame}
}

{\fullbackground[scale=0.9,page=21]{ian-motor-ctrl.pdf}
    \begin{frame}{Example: overall closed-loop transfer function}
    \end{frame}
}


{\fullbackground[scale=0.9,page=22]{ian-motor-ctrl.pdf}
    \begin{frame}{Example: overall closed-loop transfer function}
    \end{frame}
}


\begin{frame}{}
    \begin{center}
        \Large
        That's all, folks!\\[2em]
        \normalsize
        Questions:\\
        Portland Square B316 or \url{severin.lemaignan@plymouth.ac.uk} \\[1em]

        Slides:\\
        \href{https://github.com/severin-lemaignan/module-mobile-and-humanoid-robots}{\small
        github.com/severin-lemaignan/module-introduction-sensors-actuators}


    \end{center}
\end{frame}



\end{document}
