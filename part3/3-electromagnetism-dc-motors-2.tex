%!TEX program = xelatex

\documentclass[compress]{beamer}
%--------------------------------------------------------------------------
% Common packages
%--------------------------------------------------------------------------

\definecolor{links}{HTML}{663000}
\hypersetup{colorlinks,linkcolor=,urlcolor=links}

\usepackage[english]{babel}
\usepackage{pgfpages} % required for notes on second screen
\usepackage{graphicx}

\usepackage{multicol}

\usepackage{tabularx,ragged2e}
\usepackage{booktabs}

\setlength{\emergencystretch}{3em}  % prevent overfull lines
\providecommand{\tightlist}{%
  \setlength{\itemsep}{0pt}\setlength{\parskip}{0pt}}


\usetheme{hri}

% Display the navigation bullet even without subsections
\usepackage{remreset}% tiny package containing just the \@removefromreset command
\makeatletter
\@removefromreset{subsection}{section}
\makeatother
\setcounter{subsection}{1}

\makeatletter
\let\beamer@writeslidentry@miniframeson=\beamer@writeslidentry
\def\beamer@writeslidentry@miniframesoff{%
  \expandafter\beamer@ifempty\expandafter{\beamer@framestartpage}{}% does not happen normally
  {%else
    % removed \addtocontents commands
    \clearpage\beamer@notesactions%
  }
}
\newcommand*{\miniframeson}{\let\beamer@writeslidentry=\beamer@writeslidentry@miniframeson}
\newcommand*{\miniframesoff}{\let\beamer@writeslidentry=\beamer@writeslidentry@miniframesoff}
\makeatother



\newcommand{\source}[2]{{\tiny\it Source: \href{#1}{#2}}}

\usepackage{tikz}
\usetikzlibrary{mindmap,backgrounds,positioning,calc,patterns}
\usepackage{pgfplots}
\pgfplotsset{compat=newest}
\usepackage{circuitikz}

\graphicspath{{figs/}}

\title{ROCO222 \newline Intro to Sensors and Actuators}
\subtitle{Electromagnetism \& DC motor -- Part 2}

\date{}
\author{Séverin Lemaignan}
\institute{Centre for Robotics and Neural Systems\\{\bf Plymouth University}}

\begin{document}


%%%%%%%%%%%%%%%%%%%%%%%%%%%%%%%%%%%%%%%%%%%%%%%%%%%%%%%%

\licenseframe{github.com/severin-lemaignan/module-introduction-sensors-actuators}

%%%%%%%%%%%%%%%%%%%%%%%%%%%%%%%%%%%%%%%%%%%%%%%%%%%%%%%%

\maketitle

%%%%%%%%%%%%%%%%%%%%%%%%%%%%%%%%%%%%%%%%%%%%%%%%%%%%%%%%

\section{DC motors datasheets}

{\fullbackground[scale=0.9,page=2]{ian-dc-motor-datasheet.pdf}
    \begin{frame}{Maxon DC motor variants}
    \end{frame}
}

{\fullbackground[scale=0.9,page=3]{ian-dc-motor-datasheet.pdf}
    \begin{frame}{Development of permanent magnets}
    \end{frame}
}

{\fullbackground[scale=0.9,page=4]{ian-dc-motor-datasheet.pdf}
    \begin{frame}{Permanent magnets}
    \end{frame}
}

{\fullbackground[scale=0.9,page=5]{ian-dc-motor-datasheet.pdf}
    \begin{frame}{Understanding motor datasheets}
    \end{frame}
}

{\fullbackground[scale=0.9,page=6]{ian-dc-motor-datasheet.pdf}
    \begin{frame}{DC motor data}
    \end{frame}
}

{\fullbackground[scale=0.9,page=7]{ian-dc-motor-datasheet.pdf}
    \begin{frame}{Speed-voltage characteristic}
    \end{frame}
}

{\fullbackground[scale=0.9,page=8]{ian-dc-motor-datasheet.pdf}
    \begin{frame}{Speed-torque characteristic}
    \end{frame}
}

{\fullbackground[scale=0.9,page=9]{ian-dc-motor-datasheet.pdf}
    \begin{frame}{Motor values at nominal voltage}
    \end{frame}
}

{\fullbackground[scale=0.9,page=10]{ian-dc-motor-datasheet.pdf}
    \begin{frame}{Operating points}
    \end{frame}
}

{\fullbackground[scale=0.9,page=11]{ian-dc-motor-datasheet.pdf}
    \begin{frame}{Effect of changing the windings}
    \end{frame}
}

{\fullbackground[scale=0.9,page=12]{ian-dc-motor-datasheet.pdf}
    \begin{frame}{Torque constant $K_M$}
    \end{frame}
}

{\fullbackground[scale=0.9,page=13]{ian-dc-motor-datasheet.pdf}
    \begin{frame}{Torque constant $K_M$}
    \end{frame}
}

{\fullbackground[scale=0.9,page=14]{ian-dc-motor-datasheet.pdf}
    \begin{frame}{Speed constant $K_n$}
    \end{frame}
}

{\fullbackground[scale=0.9,page=15]{ian-dc-motor-datasheet.pdf}
    \begin{frame}{Nominal motor characteristics}
    \end{frame}
}

{\fullbackground[scale=0.9,page=16]{ian-dc-motor-datasheet.pdf}
    \begin{frame}{List of main motor parameters}
    \end{frame}
}

{\fullbackground[scale=0.9,page=17]{ian-dc-motor-datasheet.pdf}
    \begin{frame}{Motor thermal considerations}
    \end{frame}
}

{\fullbackground[scale=0.9,page=18]{ian-dc-motor-datasheet.pdf}
    \begin{frame}{Influence of temperature on motor operation}
    \end{frame}
}

{\fullbackground[scale=0.9,page=19]{ian-dc-motor-datasheet.pdf}
    \begin{frame}{Nominal torque and temperature}
    \end{frame}
}

{\fullbackground[scale=0.9,page=20]{ian-dc-motor-datasheet.pdf}
    \begin{frame}{Motor limits: operation ranges}
    \end{frame}
}

{\fullbackground[scale=0.9,page=21]{ian-dc-motor-datasheet.pdf}
    \begin{frame}{Short-term overload operation}
    \end{frame}
}

{\fullbackground[scale=0.9,page=22]{ian-dc-motor-datasheet.pdf}
    \begin{frame}{List of main thermal motor parameters}
    \end{frame}
}

{\fullbackground[scale=0.9,page=23]{ian-dc-motor-datasheet.pdf}
    \begin{frame}{List of main mechanical motor parameters}
    \end{frame}
}

{\fullbackground[scale=0.9,page=24]{ian-dc-motor-datasheet.pdf}
    \begin{frame}{List of main mechanical motor parameters}
    \end{frame}
}

{\fullbackground[scale=0.9,page=25]{ian-dc-motor-datasheet.pdf}
    \begin{frame}{Other specifications}
    \end{frame}
}


{\fullbackground[scale=0.9,page=26]{ian-dc-motor-datasheet.pdf}
    \begin{frame}{Motor size selection}
    \end{frame}
}


\miniframesoff
\begin{frame}[plain]
    \begin{center}
        \Large
        10 min break\\[2em]
    \end{center}
\end{frame}
\miniframeson

\section[Recap]{So far in electromagnetism...}

\begin{frame}{Recap of last lecture}
    \begin{itemize}
        \item Ampère's law
        \item Faraday's law of induction: $\displaystyle\mathcal{E} = -N \cdot \frac{d\Phi_B}{dt}$
        \item Lenz law
        \item Lorentz law
    \end{itemize}
\end{frame}

\section{Inductance}

{\fullbackground[scale=0.9,page=39]{ian-electromagnetism.pdf}
    \begin{frame}{Electrical resistance}
    \end{frame}
}

{\fullbackground[scale=0.9,page=40]{ian-electromagnetism.pdf}
    \begin{frame}{Electrical inductance}
    \end{frame}
}

{\fullbackground[scale=0.9,page=41]{ian-electromagnetism.pdf}
    \begin{frame}{Current in an LR circuit}
    \end{frame}
}

{\fullbackground[scale=0.9,page=42]{ian-electromagnetism.pdf}
    \begin{frame}{Current in an LR circuit}
    \end{frame}
}

{\fullbackground[scale=0.9,page=43]{ian-electromagnetism.pdf}
    \begin{frame}{Definition of inductance}
    \end{frame}
}

{\fullbackground[scale=0.9,page=44]{ian-electromagnetism.pdf}
    \begin{frame}{Inductance of a solenoid}
    \end{frame}
}


\begin{frame}{Consequence for motors}

  Moving a magnet in a coil induces current


    \begin{center}
        \includegraphics[width=0.6\linewidth]{image29}

    \scalebox{1.5}{Faraday's law of induction: $\displaystyle\mathcal{E} = -N \cdot \frac{d\Phi_B}{dt}$}

    \end{center}
\end{frame}

\begin{frame}{Electrical generator}

    \begin{center}
        \includegraphics[width=0.6\linewidth]{image30}
    \end{center}

\begin{itemize}

\item Speed of rotation affects voltage generated
\item So how fast will a motor rotate with applied voltage $V$?
\item Lets consider the equivalent circuit for a DC motor
\end{itemize}
\end{frame}

\begin{frame}{DC motor equivalent circuit}

    \begin{center}
        \resizebox{0.6\linewidth}{!}{
            % Based on http://texample.net/tikz/examples/induction-machine/
            \begin{circuitikz}
                \draw
                % rotor circuit
                (0,0) to [short, *-] (6,0)
                to [V, l_={EMF}] (6,2) % rotor emf

                % stator circuit
                (0,0) to [open, v^>=$V$] (0,2) % stator voltage
                to [short, *- ,i=$I$] (1,2) % stator current
                to [R, l=$R$] (3,2) % stator resistance
                to [L, l=$L$] (6,2); % leakage inductance
            \end{circuitikz}
        }
    \end{center}

\footnotesize
\begin{itemize}

\item $V$ is the applied voltage, $I$ is the drawn current
\item Resistance $R$ arises from the coil and the brushes
\item Inductance $L$ arises from the coil
\item Back EMF arises from rotation of the coil in the magnetic field
  created by the stator magnets
\item Therefore in no-load condition motor will speed up and reach steady
  state when: $V$ = back EMF
\end{itemize}

\end{frame}


\section[Diff. equations]{Differential equation of DC motor rotation}

{\fullbackground[scale=0.9,page=6]{ian-simple-newtonian-mechanics.pdf}
\begin{frame}{Moment of Inertia}
%
%Resists with opposing torque proportional to angular acceleration
%
%    \begin{columns}
%        \begin{column}{0.5\linewidth}
%            \begin{center}
%                \includegraphics[height=0.3\paperheight]{image62}
%            \end{center}
%        \end{column}
%        \begin{column}{0.5\linewidth}
%            TBD
%        \end{column}
%    \end{columns}
%where
%
%$T$ is torque in $N\cdot m$
%
%$J$ is moment of inertial in $kg \cdot m^2$
%
\end{frame}
}

\begin{frame}{DC motor dynamics}

    \begin{center}
        \includegraphics[width=0.5\linewidth]{image63}
    \end{center}

\only<1> {
\begin{itemize}

\item $R$ = Armature resistance (in ohms $\Omega$)
\item $L$ = Armature inductance (in Henrys $H$) % CHECK
\item $J$ = Moment of inertia for the motor rotor ($kg\cdot m^2$)
\item $b$ = Motor viscous friction constant (in $N\cdot m\cdot s$)
\item $K_t$ = Motor torque constant (in $\frac{N\cdot m}{A}$)
\item $K_e$ = Electromotive force constant (in $\frac{V\cdot rad^{-1}}{sec}$) % CHECK
\end{itemize}
}

    \only<2> {
    Motor torque $\tau_m$ is given by $\tau_m = K_t \cdot i(t)$

Mechanical resisting torque $\tau_r$ is given by $\tau_r = b \cdot \dot\theta + J \cdot \ddot\theta$

Under no load, $\tau_m = \tau_r$

Therefore:
\[
    K_t \cdot i(t) = b \cdot \dot\theta + J \cdot \ddot\theta
\]

}
    \only<3> {

\textbf{Kirchhoff's voltage law}: voltage across resistor, inductance and
back EMF balance applied voltage

    \[
        v(t) = i(t) \cdot R + L \cdot \frac{di}{dt} + K_e \cdot \dot\theta
    \]
}
\end{frame}

\begin{frame}{Definition of Laplace transform}

    \only<1>{
The Laplace transform is a linear operator that maps a function $f(t)$ to
$F(s)$.


Specifically:

\[
    F(s) = \mathcal{L}\{f\}(s) =  \mathcal{L}\{f(t)\} = \int^{\inf}_{0} f(t)e^{-st}dt
\]

where $s = \sigma + i\omega$

    Go from a function of a \emph{real} variable (here time $t$) to a complex
    function of a complex variable (frequency, $s$).

}

    \only<2> {

        Why bother?

        Often \textbf{simplifies the process of analyzing the behavior of the
        system}.

        For example, Laplace transformation from the time domain to the
        frequency domain \textbf{transforms differential equations into algebraic
        equations}.
    }
\end{frame}

\begin{frame}{Operations useful for solving differential equations}

\Large

\[
    \mathcal{L} \{f'(t)\} = sF(s) - f(0)
\]


\[
    \mathcal{L} \{f''(t)\} = s^2F(s) - sf(0) - f'(0)
\]


\[
    \mathcal{L} \{\int^{t}_{0}f(t)dt\} = \frac{F(s)}{s}
\]

\vspace{2em}
\small
See
    \href{https://en.wikipedia.org/wiki/Laplace_transform}{Wikipedia}
    for more properties.

\end{frame}

\begin{frame}{Solution using Laplace transformations}

\only<1>{

Taking Laplace transforms of the differential equations that describe
the motor mechanical dynamics:

\[
   K_t \cdot i(t) = b \cdot \dot\theta + J \cdot \ddot\theta
\]

becomes:

\[
    K_t \cdot I(s) = b \cdot s \cdot \theta(s) + J \cdot s^2 \cdot \theta(s)
\]

\[
    I(s) = \frac{s \cdot (b + J \cdot s) \cdot \theta(s)}{K_t}
\]

}

    \only<2> {

Taking Laplace transforms of the differential equations that describe
the motor voltages:

\[
        v(t) = i(t) \cdot R + L \cdot \frac{di}{dt} + K_e \cdot \dot\theta
\]

becomes:

\[
    V(s) = I(s) \cdot (R + L\cdot s) + K_e \cdot s \cdot \theta(s)
\]
}

    \only<3>{

Substituting:

\[
    I(s) = \frac{s \cdot (b + J \cdot s) \cdot \theta(s)}{K_t}
\]


into:

\[
    V(s) = I(s) \cdot (R + L\cdot s) + K_e \cdot s \cdot \theta(s)
\]

Eliminating current $I(s)$ and setting $K_t = K_e = K$ gives:

\[
    \theta(s) = \frac{K}{s \cdot ( (J \cdot s + b) \cdot (L \cdot s+ R) + K^2)} \cdot V(s)
\]

}

\end{frame}

\begin{frame}{Transfer function}

    \begin{columns}
        \begin{column}{0.5\linewidth}

            Result of a transfer response output position for a DC electric motor
            given its input voltage

        \end{column}
        \begin{column}{0.5\linewidth}


            \begin{center}
                \includegraphics[width=0.9\columnwidth]{image63}
            \end{center}

        \end{column}
    \end{columns}

\[
    \theta(s) = \frac{K}{s \cdot ( (J \cdot s + b) \cdot (L \cdot s+ R) + K^2)} \cdot V(s)
\]

\pause

Can differentiate this expression to get the transfer function for speed

\[
    \dot\theta(s) = \frac{K}{(J \cdot s + b) \cdot (L \cdot s + R) + K^2} \cdot V(s)
\]


\end{frame}

%%%%%%%%%%%%%%%%%%%%%%%%%%%%%%%%%%%%%%%%%%%%%%%%%%%%%%%%%%%%%%%%%%
\section{Brushless DC motors}

{\fullbackground[scale=0.9,page=2]{ian-brushless-dc-motors.pdf}
\begin{frame}{Problems of mechanical commutation}

%Can get potential difference across commutator segments
%
%\begin{itemize}
%
%\item Can get potential difference across commutator segments
%\item Commutation shorts out the commutator segments
%\item Arcing and sparkling at the brushes
%\item Brushless electronic switching solves this issue
%\end{itemize}

\end{frame}
}

{\fullbackground[scale=0.9,page=3]{ian-brushless-dc-motors.pdf}
\begin{frame}{Brushless DC Motor}

%This motor type looks like DC brushed motor turned inside out!
%
%\begin{itemize}
%
%\item This motor type looks like DC brushed motor turned inside out!
%\item In an EC (brushless) motor, commutation is performed electronically to
%  eliminate brushes
%\item The stator generally consists of several coils
%\item Current flow in the stator coils creates magnetic field
%\item This forces the permanent magnet rotor turn
%\item The rotor can be forced to rotate continuously by switching on current
%  in the stator coils in the appreciate sequence thereby generating a
%  sequenced magnetic field
%\item All brushless motors require a controller to work that must perform
%  the commutation operation
%\end{itemize}

\end{frame}
}

{\fullbackground[scale=0.9]{image81}
\begin{frame}{Typical brushless motor}

\end{frame}
}


{\fullbackground[scale=0.9,page=5]{ian-brushless-dc-motors.pdf}
\begin{frame}{How do brushless motors work?}

%\begin{itemize}
%\item Electronic commutation is used to switch current in the stator could
%  so that the rotor is forced to rotate
%\item There is often a control magnet is in line with the poles of the large
%  magnet in the motor to identify rotor angle so that the controller can
%  switch current into the appropriate coils
%\item As it turns Hall sensors are stimulated by the magnetic flux.
%\item The Hall sensors are used to tell the controller what the orientation
%  is of the magnet with respect to the three winding phases.
%\item Current in the stator coils is turned on and off in sequence creating
%  motion from pole to pole.
%\end{itemize}

\end{frame}
}

{\fullbackground[scale=0.9,page=6]{ian-brushless-dc-motors.pdf}
\begin{frame}{Block commutation}

%\begin{itemize}
%\item
%\end{itemize}
%
%\_
%
%HS3
%
%HS1
%
%HS2
%
%controller
%
%power stage
%
%(MOSFET)
%
%phase 1
%
%phase 2
%
%phase 3
%
%EC motor
%
%(magnet, winding, sensor)
%
%rotor position feedback
%
%commutation
%
%logics
%
    \note{

\textbf{On the right} we have a schematic \textbf{cross section of a
maxon EC motor} with 2 pole permanent magnet in the center, the three
phase winding and the three Hall sensors placed at 120°. For simplicity
we assume the Hall sensors to probe the power magnet directly.

\textbf{On the left} we have the \textbf{commutation electronics} which
is fed with a DC supply voltage. There is a power bridge made of 6
MOSFETs. Three of them are needed to contact the motor phases to the
positive supply voltage. The lower three MOSFETs make the contact to the
supply ground. The power bridge is controlled by a commutation logic
that evaluates the Hall sensor signals and, accordingly, switches the
power on the three motor phases.

\textbf{Comments on the animation:}

In this starting position the Hall sensors give the following signal:
HS1 has just switched to a high state, HS2 is low and HS3 is high.

The commutation logic knows that for this signal combination and
clockwise motor rotation the current must flow from phase 1 to 2 and
powers the respective two MOSFETs.

The winding produces a magnetic field and the magnetic rotor tries to
align.

After 60° the HS3 starts seeing the south pole. Its output switches to
low and the commutation logic switches the current from phase 1 to 3.
The field of the winding advances by 60° and the rotor continues to
rotate.

Again after 60° the Hall sensor pattern changes, HS2 switches to a high
output level. Accordingly the electronics commutates the current to flow
from phase 2 to 3. Again the field of the winding advances by another
60° and the rotor continues.

And so on \ldots{} . After 6 commutation intervals we are back at the
initial configuration and the rotor has accomplished one turn.

}

\end{frame}

}

{\fullbackground[scale=0.9,page=7]{ian-brushless-dc-motors.pdf}
\begin{frame}{Brushless motor for RC aircraft}

\end{frame}
}

{\fullbackground[scale=0.9,page=8]{ian-brushless-dc-motors.pdf}
\begin{frame}{Maxon EC brushless motor}

%Permanent magnet
%
%Special Winding
%
%Rotating part -- permanent magnet
%
%Hall sensors
%
%Control Magnet
%
%Case / Magnetic
%
%return

\end{frame}

}

{\fullbackground[scale=0.9,page=9]{ian-brushless-dc-motors.pdf}
\begin{frame}{Maxon EC flat brushless motor}

%Multi pole motor
%
%Flat design gives more torque as the flux is acting further from the
%centre of rotation

\end{frame}

}

{\fullbackground[scale=0.9,page=10]{ian-brushless-dc-motors.pdf}
\begin{frame}{Advantages and disadvantages of EC}

%\textbf{Brushed DC motors}
%
%\begin{itemize}
%
%\item Mechanical commutation
%\item Need periodic brush maintenance
%\item Power losses in brushes
%\item Sparking
%\item Can have noisy operation
%\item Linear torque characteristic at lower
%\item Change direction by changing voltage polarity
%\item Controller not always needed
%\end{itemize}
%
%\textbf{EC motors}
%
%\begin{itemize}
%
%\item Electronic commutation
%\item Low or no maintenance
%\item Less power loss
%\item No sparking
%\item Quieter operation
%\item More linear torque characteristic
%\item Change direction by changing switching sequence
%\item Always needs drive controller circuitry
%\item Requires sensors
%\item Higher reliability \& efficiency
%\item Stator on outside -- better for heat dissipation
%\item Longer life
%\item More expensive
%\end{itemize}

\end{frame}
}

%\section{Some other motors}
%
%\begin{frame}{Wound field motors}
%
%\begin{itemize}
%
%\item What happens if we apply AC to a permanent magnet DC motor?
%\end{itemize}
%
%\end{frame}
%
%\begin{frame}{Motor with stator winding}
%
%\end{frame}
%
%\begin{frame}{Shunt motor}
%
%\begin{itemize}
%\item Like DC motor but with electromagnet to generate static field
%\item Armature and field windings are connected in parallel
%\item Separate current through stator and armature
%\item Low Starting Torque
%\item Good Speed Regulation
%\item Used for fixed speed applications, windscreen wipers, fans
%\end{itemize}
%
%\end{frame}
%
%\begin{frame}{Shunt motor}
%
%Consider motor behavior under load:
%
%\begin{itemize}
%
%\item On application of load speed will reduce
%\item But this reduced armature EMF
%\item Therefore armature current rises
%\item Therefore torque increases
%\item So speed increases too
%\item Therefore system can do some self regulation of speed
%\item Much like permanent magnet DC motor!
%\end{itemize}
%
%\end{frame}
%
%\begin{frame}{Series motor}
%
%\begin{itemize}
%
%\item Armature and field windings are connected in series
%\item Same current goes through both
%\item High Starting Torque
%\item As the speed builds up so does the back EMF, reducing the current,
%  which causes a reduction in torque
%\item Poor Speed Regulation
%\item Used for starting heavy, industrial, high torque loads such as cranes,
%  hoists, elevators, trolleys and conveyors
%\item Cannot operate safely in an unloaded condition
%\end{itemize}
%
%\end{frame}
%
%\begin{frame}{Universal Motors}
%
%\begin{itemize}
%
%\item Series motor
%\item Uses field coils and not permanent magnets
%\item AC and DC operation
%\item As current direction changes it changes field direction on stator
%  field and also armature
%\item So always rotates in same direction independent of applied current
%  direction
%\end{itemize}
%
%\end{frame}



%%%%%%%%%%%%%%%%%%%%%%%%%%%%%%%%%%%%%%%%%%%%%%%%%%%%%%%%
%%%%%%%%%%%%%%%%%%%%%%%%%%%%%%%%%%%%%%%%%%%%%%%%%%%%%%%%
\miniframesoff
\begin{frame}{}
    \begin{center}
        \Large
        That's all, folks!\\[2em]

        \normalsize
        \textbf{Questions}:\\
        Portland Square B316 or \url{severin.lemaignan@plymouth.ac.uk} \\[1em]

        \textbf{Slides}:\\
        \href{https://github.com/severin-lemaignan/module-introduction-sensors-actuators}{\small
        github.com/severin-lemaignan/module-introduction-sensors-actuators} \\

        ...or the DLE!


    \end{center}
\end{frame}




\end{document}
