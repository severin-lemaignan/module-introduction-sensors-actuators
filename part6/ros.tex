%!TEX program = xelatex

\documentclass[compress]{beamer}
%--------------------------------------------------------------------------
% Common packages
%--------------------------------------------------------------------------

\definecolor{links}{HTML}{663000}
\hypersetup{colorlinks,linkcolor=,urlcolor=links}

\usepackage[english]{babel}
\usepackage{pgfpages} % required for notes on second screen
\usepackage{graphicx}

\usepackage{pdfpcnotes}

\usepackage{multicol}

\usepackage{tabularx,ragged2e}
\usepackage{booktabs}

\setlength{\emergencystretch}{3em}  % prevent overfull lines
\providecommand{\tightlist}{%
  \setlength{\itemsep}{0pt}\setlength{\parskip}{0pt}}


\usetheme{hri}

% Display the navigation bullet even without subsections
\usepackage{remreset}% tiny package containing just the \@removefromreset command
\makeatletter
\@removefromreset{subsection}{section}
\makeatother
\setcounter{subsection}{1}

\makeatletter
\let\beamer@writeslidentry@miniframeson=\beamer@writeslidentry
\def\beamer@writeslidentry@miniframesoff{%
  \expandafter\beamer@ifempty\expandafter{\beamer@framestartpage}{}% does not happen normally
  {%else
    % removed \addtocontents commands
    \clearpage\beamer@notesactions%
  }
}
\newcommand*{\miniframeson}{\let\beamer@writeslidentry=\beamer@writeslidentry@miniframeson}
\newcommand*{\miniframesoff}{\let\beamer@writeslidentry=\beamer@writeslidentry@miniframesoff}
\makeatother



\newcommand{\source}[2]{{\tiny\it Source: \href{#1}{#2}}}

\usepackage{tikz}
\usetikzlibrary{mindmap,backgrounds,positioning,calc,patterns}
\usepackage{pgfplots}
\pgfplotsset{compat=newest}
\usepackage{circuitikz}

\usepackage[normalem]{ulem}

\graphicspath{{figs/}}

\title{ROCO222 \newline Intro to Sensors and Actuators}
\subtitle{ROS, the Robot Operating System}

\date{}
\author{Séverin Lemaignan}
\institute{Centre for Neural Systems and Robotics\\{\bf Plymouth University}}

\begin{document}

\licenseframe{github.com/severin-lemaignan/module-introduction-sensors-actuators}

\maketitle

\miniframesoff

\miniframeson



\section{ROS and Arduino}

\begin{frame}[fragile]{ROS with the Arduino}

    \sh{rosserial} is a ROS \emph{bridge} that transparently transport ROS messages
    over a serial connection.

    \sh{rosserial_arduino} is a \sh{rosserial} client for the Arduino.
    You can install it easily:\\
    \sh{apt install ros-kinetic-rosserial ros-kinetic-rosserial-arduino}

    To make it transparently available in the Arduino IDE, you need to also
    install it as an Arduino library:

    \begin{shcode}
> cd $HOME/sketchbook/libraries
> rosrun rosserial_arduino make_libraries.py .
    \end{shcode}
\end{frame}

\imageframe{arduino-ide-ros}

\begin{frame}[fragile]{Arduino code to control a servo with ROS}
\begin{columns}
    \begin{column}{0.5\linewidth}
        
\begin{minted}[frame=none,
        linenos=true,
        fontsize=\tiny,
        numbersep=0.4em,
        xleftmargin=0.5em]{cpp}
#include <ros.h>
#include <std_msgs/UInt16.h>
#include <Servo.h> 

using namespace ros;

NodeHandle  nh;
Servo servo;

void cb( const std_msgs::UInt16& msg){
  servo.write(msg.data); // 0-180
}

Subscriber<std_msgs::UInt16> sub("servo", cb);

void setup(){
  nh.initNode();
  nh.subscribe(sub);

  servo.attach(9); //attach it to pin 9
}

void loop(){
  nh.spinOnce();
  delay(1);
}
\end{minted}
    \end{column}
    \begin{column}{0.5\linewidth}

        \small Python $\approx$equivalent:

\begin{minted}[frame=none,
               linenos=true,
               fontsize=\tiny,
               numbersep=0.4em,
               xleftmargin=0.5em]{python}
import rospy
from std_msgs.msg import UInt16

def cb(msg):
    # servo.write(msg.data)
    print(msg.data)

rospy.init_node('listener')
rospy.Subscriber("servo", UInt16, cb)
# servo.attach(9)
rospy.spin()
\end{minted}
    \end{column}
\end{columns}
\end{frame}

\begin{frame}[fragile]{Running the code}
    To use the code, from your 'master' ROS computer:

\begin{shcode}
> roscore
> rosrun rosserial_python serial_node.py /dev/ttyACM0
> rostopic pub --once servo std_msgs/UInt16 110
\end{shcode}
\end{frame}



\begin{frame}{}
    \begin{center}
        \Large
        That's all, folks!\\[2em]
        \normalsize
        Questions:\\
        Portland Square B316 or \url{severin.lemaignan@plymouth.ac.uk} \\[1em]

        Slides:\\
        \href{https://github.com/severin-lemaignan/module-introduction-sensors-actuators}{\small
        github.com/severin-lemaignan/module-introduction-sensors-actuators}


    \end{center}
\end{frame}



\end{document}
