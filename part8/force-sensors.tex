%!TEX program = xelatex

\documentclass[compress]{beamer}
%--------------------------------------------------------------------------
% Common packages
%--------------------------------------------------------------------------

\definecolor{links}{HTML}{663000}
\hypersetup{colorlinks,linkcolor=,urlcolor=links}

\usepackage[english]{babel}
\usepackage{pgfpages} % required for notes on second screen
\usepackage{graphicx}

\usepackage{pdfpcnotes}

\usepackage{multicol}

\usepackage{tabularx,ragged2e}
\usepackage{booktabs}

\usepackage[european,cuteinductors]{circuitikz}

\setlength{\emergencystretch}{3em}  % prevent overfull lines
\providecommand{\tightlist}{%
  \setlength{\itemsep}{0pt}\setlength{\parskip}{0pt}}


\usetheme{hri}

% Display the navigation bullet even without subsections
\usepackage{remreset}% tiny package containing just the \@removefromreset command
\makeatletter
\@removefromreset{subsection}{section}
\makeatother
\setcounter{subsection}{1}

\makeatletter
\let\beamer@writeslidentry@miniframeson=\beamer@writeslidentry
\def\beamer@writeslidentry@miniframesoff{%
  \expandafter\beamer@ifempty\expandafter{\beamer@framestartpage}{}% does not happen normally
  {%else
    % removed \addtocontents commands
    \clearpage\beamer@notesactions%
  }
}
\newcommand*{\miniframeson}{\let\beamer@writeslidentry=\beamer@writeslidentry@miniframeson}
\newcommand*{\miniframesoff}{\let\beamer@writeslidentry=\beamer@writeslidentry@miniframesoff}
\makeatother



\newcommand{\source}[2]{{\tiny\it Source: \href{#1}{#2}}}

\usepackage{tikz}
\usetikzlibrary{mindmap,backgrounds,positioning,calc,patterns}
\usepackage{pgfplots}
\pgfplotsset{compat=newest}
\usepackage{circuitikz}

\graphicspath{{figs/}}

\title{ROCO222 \newline Intro to Sensors and Actuators}
\subtitle{Force and Torque Sensors}

\date{}
\author{Séverin Lemaignan}
\institute{Centre for Neural Systems and Robotics\\{\bf Plymouth University}}

\begin{document}

\licenseframe{github.com/severin-lemaignan/module-introduction-sensors-actuators}

\maketitle

\miniframesoff

\begin{frame}{Today's objectives}

    \begin{itemize}
            \item Know how to measure force and torque
            \item (prep' for ROCO318 next year)
    \end{itemize}
\end{frame}

\miniframeson

%%%%%%%%%%%%%%%%%%%%%%%%%%%%%%%%%%%%%%%%%%%%%%%%%%%%%%%%%%%%%%%%%%
%%%%%%%%%%%%%%%%%%%%%%%%%%%%%%%%%%%%%%%%%%%%%%%%%%%%%%%%%%%%%%%%%%
%%%%%%%%%%%%%%%%%%%%%%%%%%%%%%%%%%%%%%%%%%%%%%%%%%%%%%%%%%%%%%%%%%

\section[Magnets]{Measuring force}

{\fullbackground[scale=0.9,page=2]{ian-force.pdf}
\begin{frame}{Definition of force}

\pnote{
}
\end{frame}
}
{\fullbackground[scale=0.9,page=3]{ian-force.pdf}
\begin{frame}{Four fundamental forces in nature}

\pnote{
}
\end{frame}
}

{\fullbackground[scale=0.9,page=4]{ian-force.pdf}
\begin{frame}{Force arises from gravity acting on a mass}

\pnote{
}
\end{frame}
}

{\fullbackground[scale=0.9,page=5]{ian-force.pdf}
\begin{frame}{Measuring force by balancing known force}

\pnote{
}
\end{frame}
}

{\fullbackground[scale=0.9,page=6]{ian-force.pdf}
\begin{frame}{Measuring force by measuring strain}

\pnote{
}
\end{frame}
}

{\fullbackground[scale=0.9,page=7]{ian-force.pdf}
\begin{frame}{Piezo-electric effect}

\pnote{
}
\end{frame}
}

{\fullbackground[scale=0.9,page=8]{ian-force.pdf}
\begin{frame}{Measurement of elastic deformation}

\pnote{
}
\end{frame}
}

{\fullbackground[scale=0.9,page=9]{ian-force.pdf}
\begin{frame}{Direct measurement of elastic deformation}

\pnote{
}
\end{frame}
}

{\fullbackground[scale=0.9,page=10]{ian-force.pdf}
\begin{frame}{Strain gauge}

\pnote{
}
\end{frame}
}

{\fullbackground[scale=0.9,page=11]{ian-force.pdf}
\begin{frame}{Wheatstone bridge circuit}

\pnote{
}
\end{frame}
}

{\fullbackground[scale=0.9,page=12]{ian-force.pdf}
\begin{frame}{Wheatstone bridge circuit}

\pnote{
}
\end{frame}
}

{\fullbackground[scale=0.9,page=13]{ian-force.pdf}
\begin{frame}{Simple cantilever load cell}

\pnote{
}
\end{frame}
}

\section{Measuring torque}

{\fullbackground[scale=0.9,page=14]{ian-force.pdf}
\begin{frame}{Measuring torque}

\pnote{
}
\end{frame}
}

{\fullbackground[scale=0.9,page=15]{ian-force.pdf}
\begin{frame}{Measuring torque using a gauge on shaft}

\pnote{
}
\end{frame}
}

{\fullbackground[scale=0.9,page=16]{ian-force.pdf}
\begin{frame}{6 DoF load cells}

\pnote{
}
\end{frame}
}

{\fullbackground[scale=0.9,page=17]{ian-force.pdf}
\begin{frame}{Inside a large 6 DoF FT}

\pnote{
}
\end{frame}
}

{\fullbackground[scale=0.9,page=18]{ian-force.pdf}
\begin{frame}{Inside a small 6 DoF FT}

\pnote{
}
\end{frame}
}

{\fullbackground[scale=0.9,page=19]{ian-force.pdf}
\begin{frame}{A signal conditioning amplifier is needed}

\pnote{
}
\end{frame}
}

\begin{frame}{}
    \begin{center}
        \Large
        That's all, folks!\\[2em]
        \normalsize
        Questions:\\
        Portland Square B316 or \url{severin.lemaignan@plymouth.ac.uk} \\[1em]

        Slides:\\
        \href{https://github.com/severin-lemaignan/module-introduction-sensors-actuators}{\small
        github.com/severin-lemaignan/module-introduction-sensors-actuators}


    \end{center}
\end{frame}



\end{document}
